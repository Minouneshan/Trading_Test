\documentclass[11pt]{article}
\usepackage[margin=1in]{geometry}
\usepackage{booktabs}
\usepackage{array}
\usepackage{multirow}
\usepackage{hyperref}
\usepackage{enumitem}
\setlist{leftmargin=1.6em, itemsep=0.3em}
\hypersetup{
  colorlinks=true,
  linkcolor=black,
  urlcolor=blue
}

\begin{document}
\begin{center}
  {\LARGE Broadcom FY2024 + FY2025 Run-Rate Due Diligence Brief}\\[4pt]
  {\large 22 November 2025}
\end{center}

\section*{Executive takeaways}
\begin{itemize}
  \item Broadcom closed FY2024 at \$51.6B in revenue (+44\% YoY) with 63\% gross margin, but VMware-driven amortization cut GAAP operating income to \$13.5B (26\% margin) and net income to \$5.9B (11\% margin), highlighting the GAAP/non-GAAP spread investors must underwrite (FY2024 Form 10-K).
  \item Mix shifted sharply toward software: infrastructure software revenue jumped 181\% YoY to \$21.5B and now represents 42\% of company sales, while subscriptions and services contributed 41\% of FY2024 revenue (FY2024 Form 10-K, Segment Note).
  \item Cash generation remained robust --- operating cash flow reached \$20.0B (39\% margin) and free cash flow \$19.4B --- supporting \$9.8B of dividends (\$2.105/share post split) and \$7.2B of buybacks even as net debt sits at roughly \$58B (Cash Flow and Capital Returns tables).
  \item Infrastructure software leverage is visible in recent filings: the August 4, 2024 Form 10-Q shows software already 44\% of quarterly revenue, while the September 4, 2025 Q3 release reported record \$16.0B revenue, \$5.2B AI semiconductor sales (+63\% YoY) and Q4 guidance of \$17.4B, implying another step-up in software attach.
  \item Balance sheet quality is defined by intangibles (\$98B goodwill and \$40.6B intangibles equal 84\% of assets) and customer concentration (top distributor = 26\% of Q3 FY2024 revenue), reinforcing the need to monitor VMware integration, deferred revenue durability and AI networking demand concentration (FY2024 Form 10-K and Q3 FY2024 Form 10-Q).
\end{itemize}

\section*{Source corpus}
\begin{itemize}
  \item Broadcom Inc. Form 10-K for the fiscal year ended November 3, 2024 (filed Dec 2024) plus accompanying Excel financial statement extract (`Broadcom\_10k\_2024.xls`).
  \item Broadcom Inc. Form 10-Q for the quarter ended August 4, 2024 (filed September 2024).
  \item "Broadcom Inc. Announces Third Quarter Fiscal Year 2025 Financial Results and Quarterly Dividend" press release dated 4 September 2025.
  \item "Semiconductor List.xlsx" peer comp sheet (for market multiples and consensus guardrails).
\end{itemize}

\section*{Hybrid platform map (Semiconductor \& Software)}
\begin{itemize}
  \item \textbf{Semiconductor Solutions} (FY2024 Form 10-K, pp. 5--7): custom ASICs for cloud AI accelerators, networking (Ethernet, InfiniBand, optical), broadband CPE/CO silicon, storage controllers, industrial/auto optoelectronics. Strength remains in bespoke silicon for hyperscalers and telcos.
  \item \textbf{Infrastructure Software}: VMware Cloud Foundation (VCF), Tanzu application platform, VeloCloud SD-WAN/SASE, application networking and security (lateral firewalling + load balancing), mainframe (CA) portfolio, enterprise security, Fibre Channel SAN automation, Arcot payment security. VMware Cloud Foundation and Private AI services provide license portability between on-prem and hyperscaler endpoints (10-K, Software Portfolio table).
  \item \textbf{Flywheel}: Custom AI silicon and Ethernet/optical attach feed massive VMware estates as customers standardize modern private clouds, then extend to SD-WAN and security. Software ARR boosts visibility, while ASIC wins defend hardware share.
\end{itemize}

\section*{FY2024 scoreboard}
\begin{table}[h!]
  \centering
  \begin{tabular}{lccc}
    \toprule
  Metric (\$B unless noted) & FY2024 & FY2023 & YoY \\
    \midrule
    Revenue & 51.6 & 35.8 & +44\% \\
    Gross margin & 32.5 & 24.7 & +32\% \\
  Operating income & 13.5 & 16.2 & -17\% \\
  Net income & 5.9 & 14.1 & -58\% \\
    Operating cash flow & 20.0 & 18.1 & +10\% \\
    Capital expenditures & 0.6 & 0.5 & +21\% \\
    Free cash flow & 19.4 & 17.6 & +10\% \\
    R\&D as \% of revenue & 18.1\% & 14.7\% & +340 bps \\
  Diluted EPS (USD, split-adjusted) & 1.23 & 3.30 & -63\% \\
    \bottomrule
  \end{tabular}
  \caption{Consolidated results from FY2024 Form 10-K (all figures in millions converted to billions).}
\end{table}

Observations:
\begin{itemize}
  \item VMware closing (Nov 2023) turned Broadcom into a 60/40 hardware/software company but dragged GAAP profitability via \$6.0B amortization embedded in cost of revenue and opex.
  \item Free cash flow margin remained 38\%, funding both elevated dividends (\$9.8B) and \$7.2B buybacks even with \$40B of new borrowings to refinance VMware-related debt (Cash Flow Statement).
  \item R\&D dollars jumped 77\% YoY to \$9.3B, lifting R\&D intensity to 18\% as management re-invests in VMware Cloud Foundation, Private AI services and networking silicon.
\end{itemize}

\section*{Revenue mix and business model}
\begin{table}[h!]
  \centering
  \begin{tabular}{lcccc}
    \toprule
  Segment (\$B) & FY2024 & FY2023 & YoY & FY2024 mix \\
    \midrule
    Semiconductor solutions & 30.1 & 28.2 & +7\% & 58\% \\
    Infrastructure software & 21.5 & 7.6 & +181\% & 42\% \\
    Total & 51.6 & 35.8 & +44\% & 100\% \\
    \bottomrule
  \end{tabular}
  \caption{Segment revenue from FY2024 Form 10-K, Note 13.}
\end{table}

\begin{itemize}
  \item Products still comprise 59\% of FY2024 revenue (\$30.4B, +9\% YoY) while subscriptions/services scaled to 41\% (\$21.2B, +168\% YoY), giving Broadcom a more software-like backlog profile (Disaggregation tables).
  \item Infrastructure software posted \$14.0B operating income (+148\% YoY) despite \$3.2B amortization, underscoring VMware's leverage relative to hardware (Segment operating results).
  \item Top distributor accounted for 26\% of Q3 FY2024 revenue and top-five end customers 35\%, so diversification hinges on VMware upsells and AI customer wins (Q3 FY2024 Form 10-Q).
\end{itemize}

\section*{Balance sheet and liquidity}
\begin{table}[h!]
  \centering
  \begin{tabular}{lcc}
    \toprule
  (\$B) & Nov 3 2024 & Oct 29 2023 \\
    \midrule
    Cash and cash equivalents & 9.3 & 14.2 \\
    Trade receivables & 4.4 & 3.2 \\
    Inventory & 1.8 & 1.9 \\
    Total current assets & 19.6 & 20.8 \\
    Current liabilities & 16.7 & 7.4 \\
    Long-term debt & 66.3 & 37.6 \\
    Total liabilities & 98.0 & 48.9 \\
    Stockholders' equity & 67.7 & 24.0 \\
    Goodwill & 97.9 & 43.7 \\
    Intangible assets, net & 40.6 & 3.9 \\
    \bottomrule
  \end{tabular}
  \caption{Condensed balance sheet from FY2024 Form 10-K.}
\end{table}

Highlights:
\begin{itemize}
  \item Goodwill plus intangibles equal 84\% of assets post-VMware; impairment sensitivity is material if VMware renewals falter.
  \item Net debt is roughly \$58B (short + long-term debt minus cash), so sustaining FCF > \$15B is key to debt paydown while maintaining a \$10B+ dividend.
  \item Deferred revenue from VMware subscriptions now funds working capital but requires high renewal rates to avoid cash cliffs.
\end{itemize}

\section*{Latest prints}
\subsection*{Q3 FY2024 Form 10-Q (quarter ended August 4, 2024)}
\begin{itemize}
  \item Net revenue \$13.1B (+47\% YoY) with semiconductor 56\% / infrastructure software 44\% mix; AI networking demand offset weaker broadband/storage, while VMware drove software growth (10-Q MD\&A).
  \item Gross margin was \$8.4B (64\% of sales) versus 69\% prior-year as amortization of VMware intangibles pressured margins.
  \item Research and development expense rose 73\% YoY on VMware headcount, signaling sustained opex investment despite cost controls.
  \item Customer concentration remained high: one distributor 26\% of quarterly revenue; aggregate top five end customers 35\%.
\end{itemize}

\subsection*{Q3 FY2025 press release (quarter ended August 3, 2025)}
\begin{itemize}
  \item Revenue hit \$16.0B (+22\% YoY) with AI semiconductor revenue \$5.2B (+63\% YoY) and Adjusted EBITDA \$10.7B (67\% margin).
  \item GAAP net income was \$4.1B (\$0.85 diluted EPS); non-GAAP net income \$8.4B (\$1.69 EPS), showing ongoing amortization drag versus cash earnings.
  \item Operating cash flow \$7.2B minus \$0.14B capex yielded \$7.0B free cash flow (44\% of revenue) --- validating the ``low capex, high FCF'' thesis even at elevated AI investment levels.
  \item Management guided Q4 FY2025 revenue to \$17.4B (+24\% YoY) with Adj. EBITDA margin steady at 67\%, implying 2H FY2025 run-rate \~\$64B+ before VMware renewals.
\end{itemize}

\section*{Strategic themes}
\begin{enumerate}
  \item \textbf{AI-custom silicon + networking}: Continued 63\% YoY AI revenue growth (Q3 FY2025) reinforces Broadcom's lock on hyperscaler custom accelerators and Ethernet/optical fabrics. Attach opportunity: each custom accelerator typically drags NICs, PCIe switches, and optics designed by Broadcom.
  \item \textbf{Private cloud modernization}: VMware Cloud Foundation, Private AI, and Tanzu give customers cloud-like operations in regulated environments. License portability plus Private AI services should translate hardware wins (AI accelerators) into software ARR.
  \item \textbf{Telco/edge consolidation}: VeloCloud SD-WAN/SASE and Telco Cloud Platform integrate with carrier capex cycles, aligning with government-funded broadband/5G programs.
  \item \textbf{Mainframe and payments resilience}: CA/Broadcom Software plus Arcot payment security form high-retention revenue that can service debt regardless of AI cycles.
\end{enumerate}

\section*{Risks and watch items}
\begin{itemize}
  \item \textbf{Intangible-heavy balance sheet}: 84\% of assets are goodwill/intangibles; any VMware underperformance could trigger impairments and covenant pressure.
  \item \textbf{Customer concentration}: Distributor at 26\% of revenue and top-five end customers at 35\% (10-Q). Delays in hyperscaler AI programs or internal silicon moves could swing quarterly revenue.
  \item \textbf{Debt and rates}: \$67B of long-term debt (weighted avg \~4.9\%) plus \$40B recently issued fixed/floating notes raises refinancing risk if rates remain high.
  \item \textbf{Integration execution}: VMware Cloud Foundation roadmap (Private AI, VCF Edge) must harmonize pricing/support; missteps risk churn and legal exposure in multi-cloud deals.
  \item \textbf{Regulatory oversight}: Broad portfolio touches telecom, data privacy, payments, and security; compliance costs and export controls (especially around AI networking) can erode margins.
\end{itemize}

\section*{Scenario outlook (management style)}
\begin{table}[h!]
  \centering
  \begin{tabular}{p{2cm}p{6.4cm}cc}
    \toprule
  Scenario & Key assumptions & FY2025 rev (\$B) & GAAP EPS (USD) \\
    \midrule
    Bear & AI accelerator orders slip to single-digit growth; VMware renewals slow to low-90\% retention; Adj. EBITDA 62\% of sales. & 60 & 1.10 \\
  Base & Q4 FY2025 guidance (\$17.4B) annualizes; AI networking holds 60-\% growth; VMware renewals stable at mid-90\%. & 64 & 1.45 \\
  Bull & AI accelerator revenue tops \$6.2B in Q4 and sustains \$25B annual run-rate; Private AI services upsell expands software gross margin. & 70 & 1.90 \\
    \bottomrule
  \end{tabular}
  \caption{Scenarios anchored on FY2024 base, Q3 FY2025 actuals, and management's Q4 FY2025 guidance.}
\end{table}

\section*{Action items}
\begin{itemize}
  \item Track VMware Cloud Foundation renewals/ARR disclosure to validate software-driven deleveraging.
  \item Monitor AI custom silicon share of wallet at top hyperscalers (look for confirmation in upcoming 10-Q customer mix commentary).
  \item Compare dividend policy versus debt pay-down each quarter to ensure \$10B+ annual capital returns remain sustainable.
  \item Refresh Semiconductor List.xlsx multiples quarterly to benchmark Broadcom's EV/EBITDA vs. NVIDIA/AMD/Marvell spreads as AI enthusiasm evolves.
\end{itemize}

\section*{Appendix: data notes}
\begin{itemize}
  \item Unless noted, FY2024 figures reference the fiscal year ended November 3, 2024 (Broadcom Form 10-K); YoY compares to the year ended October 29, 2023.
  \item Free cash flow defined as operating cash flow minus purchases of property, plant and equipment per consolidated cash flow statement.
  \item Mix percentages derived from segment and disaggregation tables in \texttt{Broadcom\_10k\_2024.xls}; customer concentration stats from Q3 FY2024 Form 10-Q.
  \item Q3 FY2025 metrics sourced from Broadcom's September 4, 2025 earnings release; scenarios incorporate this guidance plus peer growth ranges from Semiconductor List.xlsx.
\end{itemize}

\end{document}
