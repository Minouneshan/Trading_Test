\documentclass[11pt]{article}
\usepackage[margin=1in]{geometry}
\usepackage{booktabs}
\usepackage{array}
\usepackage{multirow}
\usepackage{hyperref}
\usepackage{enumitem}
\setlist{leftmargin=1.6em, itemsep=0.3em}
\hypersetup{
  colorlinks=true,
  linkcolor=black,
  urlcolor=blue
}

\begin{document}
\begin{center}
  {\LARGE Micron FY2025 + Run-Rate Due Diligence Brief}\\[4pt]
  {\large 22 November 2025}
\end{center}

\section*{Executive takeaways}
\begin{itemize}
  \item Micron exited FY2025 with revenue $37.4$B (+49\% YoY) and gross margin back to 39.8\% as HBM3E and DDR5 shipments for hyperscale AI workloads absorbed 1-gamma output (FY2025 Form 10-K).
  \item Operating income rebounded to $9.8$B (26\% margin) and GAAP EPS hit $7.59$ after two loss-making years, aided by $17.5$B of operating cash flow, modest restructuring, and $2.0$B of government incentive reimbursements (FY2025 cash flow statement).
  \item Segment mix shifted sharply toward Compute \& Networking (CMBU 36\% of sales, +257\% YoY) while Mobile (MCBU) and Auto/Embedded (AEBU) delivered steady but lower-growth contributions; NAND remains 23\% of consolidated revenue.
  \item Q3 FY2025 (quarter ended May 29, 2025) posted $9.3$B revenue (+36\% YoY), $2.2$B operating income, and $1.9$B net income (GAAP EPS $1.68$) with nine-month operating cash flow already $11.8$B (May 29, 2025 Form 10-Q).
  \item Capex held at $15.9$B with EUV/backside power-delivery buildouts in Boise, Hiroshima, and Taiwan; Micron still leans on CHIPS/EU grants (noncurrent unearned incentives $1.0$B) and selective debt ($14.6$B total) to fund the AI/HBM build.
\end{itemize}

\section*{Source corpus}
\begin{itemize}
  \item Micron Technology, Inc. Form 10-K for the fiscal year ended August 28, 2025 (filed October 2025) --- Excel extracts for consolidated statements, segment mix, technology mix, and cash flows.
  \item Micron Technology, Inc. Form 10-Q for the quarter ended May 29, 2025 (filed July 2025) --- operations, balance sheet, cash flow, and management commentary.
  \item ``Semiconductor List.xlsx'' (November 2025) --- peer valuation, growth, beta, capital structure, and sentiment indicators.
\end{itemize}

\section*{HBM-led rebound and portfolio map}
\begin{itemize}
  \item \textbf{CMBU (Compute \& Networking)}: HBM3E, DDR5/LPDDR5X for AI servers, CXL-attached memory, and networking buffers serving hyperscalers.
  \item \textbf{CDBU (Client \& Data)}: Client/server SSDs, consumer DRAM, PCIe Gen5 NVMe, and QLC-managed NAND for secondary storage.
  \item \textbf{MCBU (Mobile)}: LPDDR5X, UFS 4.0 MCPs, and on-device AI memory stacks for flagship Android and Chromebook platforms.
  \item \textbf{AEBU (Automotive \& Embedded)}: Auto-grade DRAM/NAND for ADAS/EV platforms plus industrial/IoT NOR; multiyear Tier-1 design wins maintain backlog through 2028.
  \item \textbf{Manufacturing \& incentives}: 1-gamma DRAM and 232-layer NAND ramping with EUV and high-NA pilots; U.S. CHIPS, Japan subsidies, and European IPCEI commitments underpin the $15$--$18$B annual capex envelope.
\end{itemize}

\section*{FY2025 scoreboard}
\begin{table}[h!]
  \centering
  \begin{tabular}{lccc}
    	oprule
    Metric (USD in billions) & FY2025 & FY2024 & YoY \% \\
    \midrule
    Revenue & 37.4 & 25.1 & +48.9\% \\
    Gross margin & 14.9 & 5.6 & +165\% \\
    Operating income & 9.8 & 1.3 & +650\% \\
    Net income & 8.5 & 0.8 & +997\% \\
    Operating cash flow & 17.5 & 8.5 & +106\% \\
    Capital expenditures & 15.9 & 8.4 & +89\% \\
    Free cash flow & 1.7 & 0.1 & NM \\
    R\&D expense & 3.8 & 3.4 & +10.7\% \\
    Restructure \& other & 0.04 & 0.00 & NM \\
    Gross margin & 39.8\% & 22.4\% & +1,740 bps \\
    Operating margin & 26.1\% & 5.2\% & +2,090 bps \\
    Net margin & 22.8\% & 3.1\% & +1,970 bps \\
    \bottomrule
  \end{tabular}
  \caption{Micron FY2025 consolidated results (amounts converted from millions). Free cash flow defined as operating cash flow minus additions to property, plant, and equipment.}
\end{table}

\subsection*{Observations}
\begin{itemize}
  \item Working capital released $5.6$B of cash (receivables up $2.7$B offset by inventory draw and payables rebuild) while inventory days fell below 90 as AI and auto pipelines tightened.
  \item Government incentives ($2.0$B cash plus $1.0$B deferred) cushioned capex; Micron expects incremental CHIPS agreements in Idaho and New York to backstop 2026 EUV and HBM expansions.
  \item Interest expense ($477$M) remains manageable relative to $496$M interest income given $11.9$B cash + securities; dividend ($0.52$B) stayed intact even during the downturn.
\end{itemize}

\section*{Segment and mix dynamics}
\subsection*{FY2025 segment revenue}
\begin{table}[h!]
  \centering
  \begin{tabular}{lccc}
    	oprule
    Segment & Revenue (\$B) & Mix & YoY \\
    \midrule
    CMBU & 13.5 & 36\% & +257\% \\
    CDBU & 7.2 & 19\% & +45\% \\
    MCBU & 11.9 & 32\% & +1.6\% \\
    AEBU & 4.8 & 13\% & +2.6\% \\
    \bottomrule
  \end{tabular}
  \caption{Business unit results per FY2025 Form 10-K; YoY compares against FY2024 revenue.}
\end{table}

\subsection*{Technology mix}
\begin{itemize}
  \item DRAM contributed $28.6$B (76\% of revenue) with HBM3E and DDR5/LPDDR5X nodes driving a 62\% YoY jump; Micron is sampling 1-gamma EUV nodes for 2026 shipments.
  \item NAND delivered $8.5$B (23\%) as 232-layer TLC/QLC ramps and UFS 4.0 attach improved; managed NAND for automotive/industrial remains a pricing buffer.
  \item Other (principally NOR) added $0.3$B for embedded/industrial control and secure microcontroller use cases.
\end{itemize}

\section*{Latest quarterly pulse (Q3 FY2025)}
\begin{itemize}
  \item Revenue $9.3$B (+36\% YoY) with gross margin $3.5$B (37.7\%) and operating income $2.2$B (23.3\%); net income $1.9$B ($1.68$ diluted EPS).
  \item Segment color: CMBU/HBM backlog remains supply constrained into CY2026; client/consumer rebounded but still price sensitive; mobile OEMs accelerated LPDDR5X adoption for on-device AI; AEBU backlog anchored by Level 2+ ADAS and industrial IoT gateways.
  \item Operating cash flow for the nine-month period reached $11.8$B vs. $5.1$B prior year; capex cash out was $8.9$B with $1.3$B CHIPS/foreign incentives already received year-to-date.
  \item Balance sheet at May 29, 2025: cash $10.2$B, short-term investments $0.65$B, total debt $15.5$B, equity $50.7$B.
\end{itemize}

\section*{Balance sheet, liquidity, and capital intensity}
\begin{itemize}
  \item \textbf{Liquidity}: Cash $9.6$B + short-term investments $0.7$B + long-term marketable securities $1.6$B = $11.9$B; Micron also holds a $7$B revolving credit facility and export-credit lines tied to tool purchases.
  \item \textbf{Leverage}: Debt $14.6$B (debt/equity 26.9\%) with staggered maturities; management targets investment-grade leverage while funding $15$B+ capex.
  \item \textbf{Working capital}: Receivables $9.3$B, inventories $8.4$B (down $0.5$B YoY), payables/accrued $9.6$B; current ratio 2.5x provides cushion for AI-cycle volatility.
  \item \textbf{Capex \& fabs}: $15.9$B capex concentrated on EUV-enabled DRAM, CMOS-under-array NAND, and advanced packaging; CHIPS/foreign subsidies expected to reimburse $4$--$5$B annually once milestones are certified.
  \item \textbf{Capital returns}: Dividend resumed at $0.46$ per share annual run-rate ($0.52$B cash); buybacks remain paused until leverage and fab spend normalize.
\end{itemize}

\section*{Strategic themes and catalysts}
\begin{enumerate}
  \item \textbf{HBM scale and share}: Micron ships HBM3E today with 8-Hi/12-Hi stacks; ramping 1-gamma + HBM4 in 2026 is key to locking second-source slots at NVIDIA, AMD, and custom accelerators.
  \item \textbf{AI PC and edge memory}: CDBU/MCBU benefit from LPDDR5X and UFS 4.0 attach; Windows AI PCs and Arm-based Chromebooks require higher bandwidth per system, expanding DRAM bits even without unit growth.
  \item \textbf{Automotive pipeline}: AEBU backlog exceeds $15$B through 2028; zonal architectures and Level 2+ sensor fusion double DRAM content per vehicle.
  \item \textbf{Government incentives}: U.S. CHIPS grants/loans for Idaho and New York plus Japan/Hiroshima subsidies improve ROIC and allow Micron to keep capex near $15$B without levering the balance sheet.
  \item \textbf{Node leadership}: Transition to EUV-based 1-gamma DRAM and 3D NAND with CMOS-under-array/backside power delivery aims to close the cost gap vs. Samsung \& SK hynix.
\end{enumerate}

\section*{Risks and watch items}
\begin{itemize}
  \item \textbf{HBM supply-demand}: If GPU launches slip or customers multi-source more aggressively, HBM pricing could normalize quickly, hurting CMBU margins.
  \item \textbf{China exposure}: Export controls on advanced DRAM/NAND or further Entity List actions could limit upside from Chinese cloud and handset OEMs.
  \item \textbf{Process execution}: Delays on 1-gamma EUV or 232-layer NAND CMOS-under-array yield could forfeit cost leadership in 2026.
  \item \textbf{Capex and subsidies}: Slower CHIPS disbursements or higher tool inflation could force incremental debt or reintroduce negative free cash flow.
  \item \textbf{Pricing discipline}: Client SSD and mobile memory remain highly competitive; oversupply in 2026 would compress gross margin back toward the 30\% range.
\end{itemize}

\section*{Scenario outlook}
\begin{table}[h!]
  \centering
  \begin{tabular}{p{2cm}p{6.4cm}cc}
    	oprule
    Scenario & Key assumptions & FY2026 revenue (\$B) & FY2026 EPS (USD) \\
    \midrule
    Bear & HBM orders push out, GPU launch cadence slows, and AI PC uplift slips to CY2027; pricing down mid-teens, incentives delayed. & 45 & 10.0 \\
    Base & HBM4/1-gamma ramps smoothly, AI PCs reach low-teens PC mix, auto backlog stays firm, and subsidies fund 25\% of capex. & 52 & 16.0 \\
    Bull & HBM4 wins at multiple hyperscalers, LPDDR5X becomes standard across Android flagships, auto wins accelerate, and CHIPS/Japan grants arrive early. & 60 & 20.0 \\
    \bottomrule
  \end{tabular}
  \caption{Scenario guardrails anchored on FY2025 actuals, Q3 FY2025 run-rate, and consensus inputs from Semiconductor List.xlsx (NTM revenue $56.5$B, NTM EPS $17.6$).}
\end{table}

\section*{Action items}
\begin{itemize}
  \item Track CHIPS and IPCEI award milestones plus actual cash reimbursements versus the $15$B capex run-rate.
  \item Monitor HBM capacity adds (stack heights, TSV yields, package supply) alongside NVIDIA/AMD/ASIC demand indicators.
  \item Watch AI PC launch cadence (Microsoft Copilot+, Qualcomm/AMD/Intel platforms) and associated DRAM/NAND bill of materials.
  \item Review auto Tier-1 win announcements and backlog conversion to gauge AEBU revenue visibility.
  \item Refresh valuation comps (PS NTM 4.74x, PE NTM 13.5x, beta 1.53) against NVIDIA, AMD, SK hynix on each earnings update.
\end{itemize}

\section*{Appendix: data notes}
\begin{itemize}
  \item Figures sourced from Micron FY2025 Form 10-K and Q3 FY2025 Form 10-Q spreadsheets included in the workspace; values converted from millions to USD billions and rounded to one decimal unless noted.
  \item Free cash flow defined as operating cash flow minus additions to property, plant, and equipment; leverage expressed as total debt divided by shareholders' equity.
  \item Valuation metrics, beta, EPS/revenue growth, dividend yield, and short interest pulled from ``Semiconductor List.xlsx'' (November 2025 refresh).
  \item Segment abbreviations follow Micron disclosures: CMBU (Compute \& Networking), CDBU (Client \& Data), MCBU (Mobile), AEBU (Automotive \& Embedded).
\end{itemize}

\end{document}
