\documentclass[11pt]{article}
\usepackage[margin=1in]{geometry}
\usepackage{booktabs}
\usepackage{array}
\usepackage{multirow}
\usepackage{hyperref}
\usepackage{enumitem}
\setlist{leftmargin=1.6em, itemsep=0.3em}
\hypersetup{
  colorlinks=true,
  linkcolor=black,
  urlcolor=blue
}

\begin{document}
\begin{center}
  {\LARGE AMD FY2024 + FY2025 Run-Rate Due Diligence Brief}\\[4pt]
  {\large 22 November 2025}
\end{center}

\section*{Executive takeaways}
\begin{itemize}
  \item AMD closed FY2024 at $25.8$B in revenue (+14\% YoY) with gross margin rebounding to 49.3\%, operating income $1.9$B and net income $1.6$B (6.4\% net margin) after a year of elevated amortization and MI300 launch costs (FY2024 Form 10-K).
  \item Data Center nearly doubled to $12.6$B (+94\% YoY) on MI300 accelerators, EPYC Genoa/Bergamo CPUs and adaptive NIC attach, pushing the segment to 49\% of company revenue; Client bounced 52\% on Ryzen AI notebooks while Gaming (-58\%) and Embedded (-33\%) digested inventory (Segment Note).
  \item Free cash flow reached $2.4$B on $3.0$B operating cash flow and $0.6$B capex, funding $0.86$B buybacks, $0.73$B tax-withholding repurchases, and $0.75$B debt retirement while keeping net cash roughly $2.3$B (cash/investments $5.1$B vs. debt $1.7$B) (Cash Flow and Balance Sheet).
  \item Intangibles dominate the balance sheet post-Xilinx ($24.8$B goodwill, $18.9$B acquisition intangibles) but amortization ($2.4$B in 2024) is rolling off; R\&D intensity stayed elevated at 25\% of revenue as AMD races to keep its MI300/Strix Point/Ryzen AI product cadence ahead of hyperscaler demand (Item 7).
  \item Q3 FY2025 run-rate reinforces the AI pivot: quarterly revenue $9.25$B (+36\% YoY) with Data Center $4.3$B, Client $2.75$B and Gaming $1.30$B, plus consolidated gross margin 51.7\% and opex leverage from AI PCs (September 27, 2025 Form 10-Q).
\end{itemize}

\section*{Source corpus}
\begin{itemize}
  \item Advanced Micro Devices, Inc. Form 10-K for the fiscal year ended December 28, 2024 (filed February 2025).
  \item AMD Form 10-Q for the quarter ended September 27, 2025 (filed November 2025).
  \item ``Semiconductor List.xlsx'' (updated November 2025) for peer growth, valuation, and profitability benchmarks.
\end{itemize}

\section*{Heterogeneous compute platform map}
\begin{itemize}
  \item \textbf{Data Center}: MI300X/MI300A accelerator platforms, EPYC Genoa/Bergamo/Turin CPUs, Pensando/DPUs, adaptive NICs and Xilinx Versal FPGAs position AMD across training, inference, and SmartNIC workloads.
  \item \textbf{Client \& AI PCs}: Ryzen 8000/9000-series CPUs with on-die Ryzen AI NPUs target the Windows AI PC upgrade, while Strix Point APUs meld Zen5 + RDNA3.5 + XDNA 2 for on-device generative AI.
  \item \textbf{Gaming}: RDNA3 discrete GPUs and semi-custom SoCs (Sony, Microsoft) remain the console anchor though FY2024 revenue compressed during the post-pandemic lull.
  \item \textbf{Embedded}: Xilinx adaptive compute (industrial/auto/5G) plus AMD's networking/vision catalog; growth will return as carrier and industrial capex pauses resolve.
  \item \textbf{Software/Systems}: ROCm 6, open AI software stacks, and Infinity Fabric tie the silicon portfolio together; AMD is leaning on co-packaged optics, CXL memory pooling, and modular chiplets to stay competitive versus NVIDIA and Intel Foundry services.
\end{itemize}

\section*{FY2024 scoreboard}
\begin{table}[h!]
  \centering
  \begin{tabular}{lccc}
    \toprule
    Metric (USD in billions) & FY2024 & FY2023 & YoY \\
    \midrule
    Revenue & 25.8 & 22.7 & +13.7\% \\
    Gross profit & 12.7 & 10.5 & +21.7\% \\
    Operating income & 1.9 & 0.4 & +373\% \\
    Net income & 1.6 & 0.9 & +92\% \\
    Operating cash flow & 3.0 & 1.7 & +82\% \\
    Capital expenditures & 0.6 & 0.5 & +16\% \\
    Free cash flow & 2.4 & 1.1 & +115\% \\
    R\&D expense & 6.5 & 5.9 & +9.9\% \\
    Gross margin & 49.3\% & 46.1\% & +320 bps \\
    Operating margin & 7.4\% & 1.8\% & +560 bps \\
    Net margin & 6.4\% & 3.8\% & +260 bps \\
    \bottomrule
  \end{tabular}
  \caption{Consolidated results from FY2024 Form 10-K (amounts converted from millions).}
\end{table}

\subsection*{Observations}
\begin{itemize}
  \item Operating leverage returned as amortization stabilized and MI300 shipments scaled; total amortization of acquisition-related intangibles fell to $2.4$B from $2.8$B.
  \item R\&D at $6.5$B (25\% of revenue) underscores AMD's need to maintain simultaneous CPU, GPU, DPU, NPU and adaptive roadmaps.
  \item Share-based comp plus integration investments kept opex elevated, but $2.4$B FCF enabled $2.3$B of combined debt paydown and buybacks without tapping cash balances.
\end{itemize}

\section*{Segment and mix dynamics}
\subsection*{FY2024 segment revenue}
\begin{table}[h!]
  \centering
  \begin{tabular}{lccc}
    \toprule
    Segment & Revenue (\$B) & Mix & YoY \\
    \midrule
    Data Center & 12.6 & 49\% & +94\% \\
    Client & 7.1 & 27\% & +52\% \\
    Gaming & 2.6 & 10\% & -58\% \\
    Embedded & 3.6 & 14\% & -33\% \\
    \bottomrule
  \end{tabular}
  \caption{Reportable segment breakout (FY2024 Form 10-K, Item 7).}
\end{table}

\subsection*{Latest quarterly pulse (Q3 FY2025)}
\begin{itemize}
  \item Revenue $9.25$B (+36\% YoY); Data Center $4.34$B (+22\% YoY), Client $2.75$B (+46\%), Gaming $1.30$B (+181\%), Embedded $0.86$B (-8\%).
  \item Gross profit $4.78$B (51.7\% margin) vs. $3.42$B (50.1\%) prior year; operating cash flow for the first nine months was not disclosed in detail, but YTD net revenue of $24.4$B keeps AMD on pace for a $34$B+ FY2025 run-rate.
  \item Mix shift toward MI300 and Ryzen AI NPUs is visible in the quarterly net revenue table (September 27, 2025 Form 10-Q).
\end{itemize}

\section*{Balance sheet, liquidity, and capital returns}
\begin{itemize}
  \item \textbf{Liquidity}: Cash and equivalents $3.79$B plus short-term investments $1.35$B fund $5.1$B of on-hand liquidity; total current assets $19.0$B vs. current liabilities $7.3$B (2.6x current ratio).
  \item \textbf{Leverage}: Long-term debt $1.72$B following repayment of the 2.95\% $750$M notes; net cash position \textasciitilde{}$2.3$B even after $0.86$B buybacks and $0.73$B tax-withholding repurchases.
  \item \textbf{Working capital}: Inventories $5.73$B (+32\% YoY) as AMD staged MI300 and Ryzen AI builds; accounts receivable $6.19$B reflect hyperscaler concentration.
  \item \textbf{Intangibles}: Goodwill $24.8$B and acquisition intangibles $18.9$B remain 63\% of assets; amortization ($2.4$B) and future renewals are key for GAAP vs. cash earnings reconciliation.
  \item \textbf{Capital allocation}: $0.64$B capex prioritized MI300 capacity, test equipment, and design centers; share repurchase authorization still has \textgreater{}$6$B of capacity.
\end{itemize}

\section*{Strategic themes and catalysts}
\begin{enumerate}
  \item \textbf{AI accelerator land grab}: MI300X ramp plus MI300A for CDNA-based supercomputers positions AMD as the only credible GPU alternative to NVIDIA for 2025 AI clusters; HBM supply and CoWoS availability remain gating factors.
  \item \textbf{AI PC refresh}: Microsoft Copilot+ and Windows AI PC requirements should pull Ryzen AI/Strix into 2025 notebook cycles; OEM adoption cadence will dictate Client margin recovery.
  \item \textbf{Xilinx leverage}: Embedding Versal and RFSoC IP into Data Center SmartNICs and automotive platforms should stabilize Embedded after the current digestion.
  \item \textbf{Adaptive infrastructure}: CXL memory pooling, Infinity architecture, and chiplet roadmaps reinforce AMD's ``modular'' differentiation vs. monolithic Intel offerings.
  \item \textbf{Software ecosystem}: ROCm 6, open-source AI frameworks, and partnerships with PyTorch/ONNX/Triton aim to narrow CUDA's moat; developer momentum remains the key KPI.
\end{enumerate}

\section*{Risks and watch items}
\begin{itemize}
  \item \textbf{Hyperscaler concentration}: Top cloud customers drive the MI300 ramp; any internal accelerator success or qualification delays could swing quarterly revenue materially.
  \item \textbf{HBM/packaging supply}: MI300 output depends on TSMC CoWoS and HBM3/3E supply; bottlenecks would cap Data Center upside.
  \item \textbf{Embedded and Gaming digestion}: Console and industrial demand may stay weak into 2025, weighing on mix and margin.
  \item \textbf{Intangible-weighted balance sheet}: $43.7$B of goodwill/intangibles plus $61.4$B of APIC leaves AMD exposed to impairment if future cash flows lag.
  \item \textbf{Competitive response}: NVIDIA's Blackwell, Intel's Gaudi/AI PC push, and custom silicon from hyperscalers could compress pricing.
\end{itemize}

\section*{Scenario outlook}
\begin{table}[h!]
  \centering
  \begin{tabular}{p{2cm}p{6.4cm}cc}
    \toprule
    Scenario & Key assumptions & FY2025 revenue (\$B) & FY2025 EPS (USD) \\
    \midrule
    Bear & HBM/CoWoS supply limits MI300 deliveries, AI PC adoption slips to late 2026, Embedded recovery deferred; gross margin \textless{}50\%. & 32 & 2.80 \\
    Base & MI300 capacity doubles sequentially, AI PC mix hits mid-teens of Client shipments, Embedded stabilizes by 2H25. & 37 & 4.10 \\
    Bull & Hyperscalers lock multi-year MI350 contracts, Ryzen AI PCs reach 25\% mix, Embedded returns to $5$B run-rate; gross margin \textgreater{}53\%. & 45 & 5.50 \\
    \bottomrule
  \end{tabular}
  \caption{Scenario guardrails anchored on FY2024 actuals, Q3 FY2025 run-rate, and consensus from Semiconductor List.xlsx.}
\end{table}

\section*{Action items}
\begin{itemize}
  \item Track MI300/HBM capacity updates from TSMC, SK hynix, and AMD supply chain disclosures.
  \item Monitor Windows AI PC launch cadence and OEM announcements to gauge Client upside.
  \item Watch Embedded bookings (industrial/auto) for signs of digestion ending; Xilinx backlog should translate to revenue by late 2025.
  \item Compare AMD valuation (PS NTM 9.4x, PE NTM 39.6x) vs. NVIDIA, Qualcomm, and Broadcom to judge upside vs. execution risk.
  \item Refresh data points once FY2025 10-K and MI350 disclosures arrive to validate scenario EPS.
\end{itemize}

\section*{Appendix: data notes}
\begin{itemize}
  \item Financial figures pulled from AMD FY2024 Form 10-K (amounts reported in millions) and AMD Q3 FY2025 Form 10-Q; conversions to USD billions are rounded to one decimal place.
  \item Free cash flow defined as net cash provided by operating activities minus purchases of property and equipment.
  \item Segment percentages computed from FY2024 segment revenue table; quarterly mix derived from Q3 FY2025 segment disclosure.
  \item Valuation metrics sourced from ``Semiconductor List.xlsx'' (updated November 2025): Market cap $\sim$\$404B, PS NTM 9.37x, PE NTM 39.6x, net margin LTM 19\%.
\end{itemize}

\end{document}
