\documentclass[11pt]{article}
\usepackage[margin=1in]{geometry}
\usepackage{booktabs}
\usepackage{array}
\usepackage{multirow}
\usepackage{hyperref}
\usepackage{enumitem}
\setlist{leftmargin=1.6em, itemsep=0.3em}
\hypersetup{
  colorlinks=true,
  linkcolor=black,
  urlcolor=blue
}

\begin{document}
\begin{center}
  {\LARGE TSMC FY2024 + FY2025 Run-Rate Due Diligence Brief}\\[4pt]
  {\large 22 November 2025}
\end{center}

\section*{Executive takeaways}
\begin{itemize}
  \item TSMC exited FY2024 with NT\$2.89 trillion (US\$88.3B) in revenue (+34\% YoY) and 56\% gross margin, driving NT\$1.16 trillion (US\$35.3B) in net income (40\% net margin) and NT\$1.83 trillion (US\$55.7B) in operating cash flow (FY2024 Form 20-F, Item 5).
  \item High Performance Computing (HPC) platforms surged to 51\% of revenue (+9ppt YoY) while smartphones rebounded to 35\%; together they explain \textgreater{}80\% of the 2024 top-line growth, confirming that AI accelerators and flagship phones remain the core demand rails (Item 4, ``Markets and Customers'').
  \item Balance sheet quality improved: cash and marketable securities reached roughly US\$64.9B, current ratio was 2.36x, and net cash stood near NT\$1.11 trillion even after NT\$958B of capital expenditures concentrated on 3nm/2nm ramps (Items 5 and 11).
  \item Geographic mix is still dominated by North America (\textasciitilde{}69\% of revenue), but management is de-risking with Arizona (three fabs), Kumamoto (two JASM fabs) and Dresden (ESMC) --- each paired with large government grants (U.S. CHIPS Act up to US\$6.6B + US\$5B loans; Germany up to EUR5B; Japan up to JPY476B) (Item 4 and subsidy disclosures).
  \item Customer concentration intensified: the top ten customers represented 76\% of 2024 revenue, largest customer 22\%, second largest 12\%; sustaining deep co-development across nodes and advanced packaging (CoWoS, SoIC) is critical to defend share as hyperscalers pursue dual foundry strategies (Risk Factor, Item 3).
\end{itemize}

\section*{Source corpus}
\begin{itemize}
  \item Taiwan Semiconductor Manufacturing Company Limited Form 20-F for the year ended 31 December 2024 (filed 1 April 2025).
  \item TSMC October 2025 earnings release and conference-call deck (Q3 2025) for forward-looking utilization, capex and packaging commentary.
  \item "Semiconductor List.xlsx" (updated Nov 2025) for peer growth, margin, and valuation guardrails.
  \item U.S. Department of Commerce CHIPS Act fact sheet (April 2024) and ESMC/JASM subsidy announcements for incentive terms referenced in Item 4.
\end{itemize}

\section*{Foundry operating model snapshot}
\begin{itemize}
  \item \textbf{Process leadership:} Volume 3nm (N3E) production underway; 2nm (N2) risk production in 2024 with volume slated for 2025, and 1.4nm-class (A14) technology roadmap disclosed for late-decade rollouts (Item 4, Technology).
  \item \textbf{Platform overlays:} HPC covers AI accelerators, CPUs/GPUs, networking ASICs and custom silicon, benefitting from CoWoS/SoIC packaging; smartphones lean on N3E/N4P nodes with integrated RF and power management; automotive focuses on 28nm and specialty nodes + advanced packaging for ADAS.
  \item \textbf{Manufacturing footprint:} 14 fabs in Taiwan (Hsinchu, Taichung, Tainan), plus TSMC Arizona (Fab 21 Phase 1 and 2 under construction, third fab announced), Japan's JASM (Fab 1 22/28nm ramping 2024, Fab 2 adding 6/5/4nm by 2027), and Germany's ESMC (planned 300mm fab for 28/22/16/12nm nodes by 2027).
  \item \textbf{Supply chain capacity:} 2024 capex split 70\% advanced nodes (7nm and below), 20\% specialty (28/22/16/12nm, RF, embedded, power), 10\% advanced packaging and photomask; CoWoS output targeted to more than double between 2023 and 2026.
  \item \textbf{Customer structure:} Hundreds of fabless/IDM partners, but hyperscaler/system companies designing in-house silicon dominate the growth; TSMC continues to co-invest in design-technology co-optimization (DTCO) and system-technology co-optimization (STCO) to defend this base.
\end{itemize}

\section*{FY2024 scoreboard}
\begin{table}[h!]
  \centering
  \begin{tabular}{lccc}
    \toprule
    Metric & FY2024 (NT\$B) & FY2024 (US\$B) & YoY \\
    \midrule
    Net revenue & 2,894.3 & 88.3 & +33.9\% \\
    Gross profit & 1,624.9 & 49.5 & +24\% (est.) \\
    Income from operations & 1,322.0 & 40.3 & +26\% (est.) \\
    Net income attributable to shareholders & 1,158.4 & 35.3 & +36.0\% \\
    Operating cash flow & 1,826.0 & 55.7 & n/a \\
    Capital expenditures (PP\&E) & 958.0 & 29.2 & n/a \\
    Free cash flow & 868.0 & 26.5 & n/a \\
    R\&D expense & 204.2 & 6.2 & +11.9\% \\
    Current ratio & 2.36x & -- & +20 bps \\
    Net cash (cash + equivalents -- total debt) & 1,110.0 & 34.4 & n/a \\
    \bottomrule
  \end{tabular}
  \caption{Consolidated figures from FY2024 Form 20-F (Note 3 conversion rate NT\$32.79 = US\$1).}
\end{table}

Key observations:
\begin{itemize}
  \item Operating margin reached 45.7\% despite inventory digestion in consumer end markets, reflecting mix toward HPC wafers and advanced packaging value-add.
  \item Free cash flow margin was 30\%, enabling NT\$732B cash dividends plus the continuation of quarterly capital returns without tapping debt markets.
  \item Cash comprised roughly 69\% of current assets; receivables and inventories remained well below 90 days, underscoring disciplined working-capital control.
\end{itemize}

\section*{Platform and geography mix}
\subsection*{Platform revenue split}
\begin{table}[h!]
  \centering
  \begin{tabular}{lcc}
    \toprule
    Platform & FY2024 mix & YoY delta \\
    \midrule
    High Performance Computing & 51\% & +9 ppt \\
    Smartphone & 34.7\% & -3 ppt \\
    Internet of Things & 5.7\% & -2 ppt \\
    Automotive & 4.8\% & +1 ppt \\
    Digital Consumer Electronics & 1.7\% & -1 ppt \\
    Others & 2.1\% & -4 ppt \\
    \bottomrule
  \end{tabular}
  \caption{Item 4 ``Markets and Customers" platform disclosure (revenues recognized by customer platform).}
\end{table}

\subsection*{Geographic revenue split}
\begin{table}[h!]
  \centering
  \begin{tabular}{lcc}
    \toprule
    Region (customer HQ) & FY2024 mix & Commentary \\
    \midrule
    North America & 69\% & Driven by hyperscalers, CPU/GPU vendors, networking OEMs. \\
    Asia Pacific (ex-Japan/China) & 12\% & Android SoC vendors and fabless ASIC houses. \\
    China & 11\% & Recovering smartphone and datacenter designs on older nodes. \\
    EMEA & 6\% & Auto and industrial silicon (notably ADAS). \\
    Japan & 2\% & Automotive/industrial, uplift from JASM partnerships. \\
    \bottomrule
  \end{tabular}
  \caption{Geographic breakdown per Item 4 ``Markets and Customers".}
\end{table}

Takeaways:
\begin{itemize}
  \item Revenue growth in 2024 was concentrated in North America (NT\$+561B, +38\% YoY) and Asia Pacific (+NT\$109B, +63\%), reflecting AI compute waves and Android recovery.
  \item Automotive remains a small base but fastest growing (+33\% YoY) as OEMs shift ADAS/EV controllers to in-house ASICs co-developed with TSMC.
  \item China mix stabilized at low-teens due to export controls and preference for legacy nodes; risk mitigated via specialty capacity in Nanjing and global fabs.
\end{itemize}

\section*{Cash, balance sheet, and liquidity}
\begin{itemize}
  \item \textbf{Liquidity:} Cash and equivalents of NT\$2.13 trillion (US\$64.9B); financial assets at amortized cost add another NT\$364B, supporting \textgreater{}NT\$1.7T of readily available liquidity (Item 11).
  \item \textbf{Debt:} Corporate bonds of NT\$928B (US\$28.3B) and long-term loans of \textless{}NT\$35B imply net cash \textgreater{}NT\$1.11T; weighted average coupon remains below 1.5\% given Taiwan-dollar issuance.
  \item \textbf{Working capital:} Days sales outstanding \textless{}40; inventories held at \textless{}90 days thanks to flexible fab loading and close demand collaboration with hyperscalers.
  \item \textbf{Capital returns:} Cash dividends totaled NT\$732B (NT\$3.00/share quarterly cadence). Share count remained stable; treasury allocation focuses on dividends rather than buybacks.
\end{itemize}

\section*{Capacity, capex, and subsidy tracker}
\begin{itemize}
  \item \textbf{Arizona (Fab 21):} Agreements with the U.S. Department of Commerce provide up to US\$6.6B in direct CHIPS funding and up to US\$5B in federal loans. Fab 1 (N4/N5) targets volume in 2025, Fab 2 (N3/N2) scheduled for 2027, and Fab 3 (announced April 2024) will support 2nm and 1.4nm nodes later in the decade.
  \item \textbf{Japan (JASM):} Kumamoto Fab 1 (22/28nm specialty) starts shipments in 1H24; Fab 2 (with Sony, Denso, Toyota) will add 12/16nm and 6/5nm capacity by 2027. Japanese government subsidies up to JPY476B cover both phases.
  \item \textbf{Germany (ESMC):} December 2024 agreement grants up to EUR5B to build a 300mm fab in Dresden with Bosch, Infineon, and NXP participation; focus on 28/22/16/12nm automotive and industrial nodes, first production targeted for 2027.
  \item \textbf{Advanced packaging:} CoWoS output expected to triple by 2026 with new modules in Zhunan and planned Arizona packaging; SoIC capacity expanded for backside power and high-bandwidth memory stacking demanded by AI accelerators.
  \item \textbf{Capex outlook:} 2025 spend guided at US\$28--32B, again weighted \textgreater{}70\% toward advanced nodes, as TSMC accelerates N2 and A14 while bringing backside power (N2P) into risk production.
\end{itemize}

\section*{Risks and watch items}
\begin{itemize}
  \item \textbf{Customer concentration:} Top-10 customers = 76\% of revenue; largest (likely Apple) = 22\%, second largest (major GPU vendor) = 12\%. Any insourcing or dual-sourcing move could quickly pressure utilization.
  \item \textbf{Geopolitical exposure:} Majority of fabs remain in Taiwan; R.O.C. geopolitical tension, energy policy, or natural disasters could disrupt output despite geographic diversification.
  \item \textbf{Subsidy compliance:} CHIPS, Japanese, and German grants contain ``guardrails'' restricting expansion in countries of concern and joint R\&D. Breaches could trigger clawbacks or delayed disbursements.
  \item \textbf{Talent and supply chain:} Rapid global expansion (U.S., Japan, Germany) requires localized labor, procurement, and IT systems. Cultural integration issues have already been cited as ramp risks.
  \item \textbf{Advanced packaging bottlenecks:} CoWoS/SoIC demand outstrips supply; any delay in equipment deliveries (e.g., high-density RDL tools) limits AI wafer monetization.
\end{itemize}

\section*{Scenario outlook (management-style)}
\begin{table}[h!]
  \centering
  \begin{tabular}{p{2cm}p{6.4cm}cc}
    \toprule
    Scenario & Key assumptions & FY2025 revenue (US\$B) & FY2025 EPS (NT\$) \\
    \midrule
    Bear & AI wafer pull-ins pause; smartphone units flat; N3/N2 ramps slip one quarter; utilization averages mid-70\%. & 80 & 32 \\
    Base & HPC demand remains robust, N3 ramps to \textgreater{}80\% utilization, smartphones grow mid-single digits, IoT/auto steady. & 92 & 38 \\
    Bull & Third-wave AI accelerators saturate CoWoS, N2 risk production monetizes late 2025, automotive silicon doubles via ESMC/JASM support. & 100 & 43 \\
    \bottomrule
  \end{tabular}
  \caption{Revenue outcomes translated from management commentary and peer demand signals (Semiconductor List.xlsx consensus FY1/FY2). EPS approximations assume 25.9B shares outstanding and 37\% payout ratio.}
\end{table}

\section*{Action items and monitoring}
\begin{itemize}
  \item \textbf{Track CoWoS/SoIC loadings:} Validate that planned packaging expansions in Zhunan and Arizona keep pace with AI accelerator order books; bottlenecks flow directly into HPC revenue.
  \item \textbf{Monitor subsidy milestones:} Confirm drawdown schedules, compliance guardrails, and capex reimbursements for CHIPS/JASM/ESMC programs each quarter.
  \item \textbf{Watch customer diversification:} Request disclosure on the share of revenue from system companies vs. fabless customers; monitor any rise in dual-sourcing to Samsung/Intel Foundry.
  \item \textbf{Follow capex discipline:} Compare announced 2025 capex bands with actual cash outflows; ensure returns justify expanded global footprint amid potential demand volatility.
  \item \textbf{Update peer comps quarterly:} Refresh "Semiconductor List.xlsx" for valuation spreads vs. NVIDIA, AMD, Broadcom, and Samsung Foundry initiatives.
\end{itemize}

\section*{Appendix: data notes}
\begin{itemize}
  \item USD conversions follow Note 3 (NT\$32.79 = US\$1). NT\$ figures expressed in billions unless otherwise noted.
  \item Free cash flow defined as net cash provided by operating activities minus purchases of property, plant and equipment (cash flow statement, Item 5).
  \item Platform/region percentages reflect revenue recognized by customer headquarters and platform classification, not necessarily the physical shipment destination.
  \item Scenario EPS uses diluted weighted-average shares from 2024 (25.9B) and assumes a normalized tax rate of 10.5\% consistent with 2024 effective tax disclosure.
\end{itemize}

\end{document}
