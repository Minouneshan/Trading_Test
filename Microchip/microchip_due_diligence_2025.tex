\documentclass[11pt]{article}
\usepackage[margin=1in]{geometry}
\usepackage{booktabs}
\usepackage{array}
\usepackage{multirow}
\usepackage{hyperref}
\usepackage{enumitem}
\setlist{leftmargin=1.6em, itemsep=0.3em}
\hypersetup{
  colorlinks=true,
  linkcolor=black,
  urlcolor=blue
}

\begin{document}
\begin{center}
  {\LARGE Microchip FY2025 + Run-Rate Due Diligence Brief}\\[4pt]
  {\large 24 November 2025}
\end{center}

\section*{Executive takeaways}
\begin{itemize}
  \item FY2025 revenue collapsed 42\% to $4.40$B as the broad industrial/auto inventory correction starved orders, yet gross margin held 56\% and operating cash still reached $0.9$B (20\% of sales) even with fabs running well below capacity (Microchip Form 10-K).
  \item Product/geography mix stayed diversified: mixed-signal MCUs 51\% of sales, analog 26\%, and Asia 50\%/Americas 30\%/Europe 20\% --- with distributors still 45\% of revenue and Arrow at 10\%; distributor days fell to 33 while Microchip's own inventories ballooned to 251 days, forcing $132$M of under-absorption expense.
  \item Capital returns exceeded earnings: $976$M of common dividends and $96$M buybacks vs. $772$M free cash flow, funded via commercial paper, senior notes, and a new $1.45$B 7.5\% mandatory convertible preferred issuance; leverage covenants were relaxed (5.5x) while the recovery plan reduces inventory.
  \item Q2 FY2026 (Sep-25 quarter) finally showed sequential revenue growth (+4\% QoQ) and a return to modest profitability ($0.03$ diluted EPS to common) while operating cash of $364$M year-to-date came almost entirely from inventory drawdown (Form 10-Q filed 6 November 2025).
  \item Shares around \$55 (approx. \$30B market cap) trade at 5.8x NTM sales, 38x FY2026 EPS, a 3.3\% dividend yield, beta 1.47, and 80\% debt-to-equity --- a recovery/mean-reversion bet that still carries tax/audit disputes (IRS, Malaysia, Germany) and elevated payout commitments (Semiconductor List.xlsx).
\end{itemize}

\section*{Source corpus}
\begin{itemize}
  \item Microchip Technology Incorporated Form 10-K for FY2025 (filed 6 May 2025) --- consolidated statements, product/geography mix, cash flow, capital returns, CHIPS/expansion commentary.
  \item Microchip Technology Incorporated Form 10-Q for Q2 FY2026 (quarter ended 30 September 2025, filed 6 November 2025) --- quarterly P\&L, cash flow, liquidity discussion, and recovery plan updates.
  \item ``Semiconductor List.xlsx'' (November 2025 refresh) --- price, market cap, consensus revenue/EPS, valuation multiples, leverage, beta, volatility, and sentiment datapoints.
\end{itemize}

\section*{Embedded control portfolio map}
\begin{itemize}
  \item \textbf{Mixed-signal MCUs \\ \\ MPUs}: 8/16/32-bit PIC \\ AVR families, dsPIC motor-control and smart-energy controllers, TrustFLEX security modules, and broad development ecosystem.
  \item \textbf{Analog, timing, connectivity}: Interface, power-management, mixed-signal, RF, Ethernet PHY/switching, and timing solutions that attach to Microchip MCUs or serve standalone sockets (industrial/auto).
  \item \textbf{Other \\ FPGA/IP/legacy}: PolarFire FPGAs, licensing/royalty revenue (SuperFlash and IP sales), aerospace/defense ASICs, memory, and manufacturing services (foundry, assembly/test) that smooth utilization.
\end{itemize}

\section*{FY2025 scoreboard}
\begin{table}[h!]
  \centering
  \begin{tabular}{lccc}
    \toprule
    Metric (USD in billions) & FY2025 & FY2024 & YoY \\
    \midrule
    Revenue & 4.40 & 7.63 & -42.3\% \\
    Gross profit & 2.47 & 5.00 & -50.6\% \\
    Operating income & 0.30 & 2.57 & -88.5\% \\
    Net income & -0.00 & 1.91 & -100\% \\
    Operating cash flow & 0.90 & 2.89 & -68.9\% \\
    Capital expenditures & 0.13 & 0.29 & -55.8\% \\
    Free cash flow & 0.77 & 2.61 & -70.4\% \\
    R\&D expense & 0.98 & 1.10 & -10.4\% \\
    Dividends paid & 0.98 & 0.91 & +7.0\% \\
    Share repurchases & 0.10 & 0.98 & -90.2\% \\
    Gross margin & 56.1\% & 65.4\% & -930 bps \\
    Operating margin & 6.7\% & 33.7\% & -2,700 bps \\
    Net margin & -0.0\% & 25.0\% & -2,500 bps \\
    \bottomrule
  \end{tabular}
  \caption{Microchip FY2025 consolidated results (millions converted to USD billions). Free cash flow defined as operating cash flow minus capital expenditures.}
\end{table}

\subsection*{Observations}
\begin{itemize}
  \item Inventory digestion, lower factory utilization, and $259$M of interest expense drove the near-break-even bottom line despite still-solid gross margin; $490.9$M of amortization and $79.2$M of restructuring weighed further.
  \item Working-capital swing was mixed: accounts receivable fell $454$M and inventories $32$M, but accrued liabilities and LTSA deposit refunds consumed $346$M and $179$M of cash, respectively.
  \item R\&D intensity jumped to 22\% of sales as management maintains MCU/analog roadmap funding during the downturn, while SG\&A fell 16\% YoY through comp resets and travel cuts.
  \item Dividends continued unabated ($0.455$ quarterly) even as free cash flow shrank, highlighting reliance on external funding and the need for careful covenant management.
\end{itemize}

\section*{Mix snapshots}
\subsection*{Product mix (FY2025)}
\begin{table}[h!]
  \centering
  \begin{tabular}{lccc}
    \toprule
    Product line & Revenue ($\!$B) & Mix & YoY \\
    \midrule
    Mixed-signal microcontrollers & 2.25 & 51.1\% & -47.3\% \\
    Analog & 1.16 & 26.3\% & -42.6\% \\
    Other (FPGA/IP/memory/services) & 0.99 & 22.6\% & -26.1\% \\
    Total & 4.40 & 100\% & -42.3\% \\
    \bottomrule
  \end{tabular}
  \caption{Product-line disclosure from FY2025 Form 10-K.}
\end{table}

\subsection*{Geographic mix (FY2025)}
\begin{table}[h!]
  \centering
  \begin{tabular}{lccc}
    \toprule
    Geography & Revenue ($\!$B) & Mix & YoY \\
    \midrule
    Americas & 1.33 & 30.2\% & -40.2\% \\
    Europe & 0.88 & 19.9\% & -52.6\% \\
    Asia & 2.20 & 49.9\% & -38.4\% \\
    Total & 4.40 & 100\% & -42.3\% \\
    \bottomrule
  \end{tabular}
  \caption{Geographic sales split per FY2025 Form 10-K; substantially all invoiced in USD.}
\end{table}

\subsection*{Channel and operations notes}
\begin{itemize}
  \item Distributors represented 45\% of FY2025 revenue (Arrow 10\%); distributor inventory dropped to 33 days from 41, yet Microchip's own inventory days rose to 251 as fabs ran at reduced loadings.
  \item About 64\% of FY2025 sales shipped from external foundries while 67\% of assembly and test was internal; Fab 2 (Tempe) was shut in May 2025 and most expansion spending is paused through FY2026.
  \item LTSA deposits collected during the shortage period are now being refunded as customers trim commitments, decreasing deferred revenue and cash in FY2025 and FY2026.
\end{itemize}

\section*{Latest quarterly pulse (Q2 FY2026)}
\begin{itemize}
  \item \textbf{P\&L}: Net sales $1.14$B (-2\% YoY, +4\% QoQ), gross profit $638$M (55.9\% margin), operating income $88.9$M (7.8\%), GAAP net income $41.7$M; after $27.8$M preferred dividends, diluted EPS to common was $0.03$.
  \item \textbf{Six-month view}: Revenue $2.22$B (-7.9\% YoY), operating income $121$M (5.5\% margin), net loss to common $(32.5)$M as interest expense ($114$M) and opex ($1.09$B) stayed elevated.
  \item \textbf{Cash flow}: Operating cash $363.7$M hinged on a $201$M inventory release, $56$M AR collection, and $26$M payables outflow; capex $54.4$M and LTSA/tax outflows left cash down $535$M since March.
  \item \textbf{Balance sheet}: Cash $236.8$M, net receivables $746$M, inventory $1.10$B; long-term debt $5.38$B (commercial paper outstanding fluctuating, $175$M at FY2025 year-end but paid down mid-2026), total equity $6.70$B; current ratio 2.25x, but heavy dividend/interest obligations constrain flexibility.
  \item \textbf{Management commentary}: Recovery plan continues to prioritize inventory reduction, paused factory additions, and selective customer allocation; Americas and Asia posted sequential growth while Europe remains soft; LTSA refunds and softer automotive demand remain headwinds.
\end{itemize}

\section*{Balance sheet, liquidity, and capital returns}
\begin{itemize}
  \item \textbf{Liquidity}: Cash $0.77$B at FY2025 end (down to $0.24$B by September), $2.99$B current assets vs. $1.16$B current liabilities (2.6x). Revolver capacity $2.25$B plus commercial paper program ($2.25$B limit) backstops working capital.
  \item \textbf{Leverage}: Total debt $5.66$B with maturities spanning 2025--2030 senior notes plus CP; net leverage covenants temporarily relaxed (up to 6.25x in September 2025) to absorb earnings trough. Debt-to-equity around 0.8x.
  \item \textbf{Capital returns}: $975.7$M FY2025 common dividends ($0.455$ quarterly), $2.2$M preferred dividends, $96.5$M buybacks (1M shares) leaving $1.56$B authorization. Series A preferred shares convert before 2028 and currently add $55.6$M annual dividend burden.
  \item \textbf{Capex \\ CHIPS}: FY2025 capex $126$M and FY2026 plan ``at or below $100$M'' while awaiting a $162$M preliminary CHIPS Act grant for two U.S. fabs (not yet finalized). Government incentives reduced PP\&E carrying value by $46$M at FY2025.
\end{itemize}

\section*{Strategic themes and catalysts}
\begin{enumerate}
  \item \textbf{Inventory normalization}: Sequential demand uptick plus shrinking distributor days should enable utilization recovery and gross-margin rebuild once customer destocking ends.
  \item \textbf{MCU + analog attach}: Broad PIC/AVR plus power/timing portfolio positions Microchip to capture electrification, industrial automation, and connectivity sockets when industrial OEMs resume ordering.
  \item \textbf{PolarFire FPGA expansion}: Mid-range, power-efficient FPGAs and aerospace/defense sockets offer incremental growth with less competition from high-end AI accelerators.
  \item \textbf{Operational discipline}: Paused expansions, Fab 2 closure, and selective outsourcing provide levers to restore mid-60\% gross margins as loadings normalize.
  \item \textbf{Potential CHIPS grants}: Finalizing $162$M in grants and utilizing the 25\% investment tax credit could offset U.S. capex and bolster domestic manufacturing credibility with auto/defense customers.
\end{enumerate}

\section*{Risks and watch items}
\begin{itemize}
  \item \textbf{Macro/industrial demand}: A prolonged industrial/auto slowdown or renewed distributor destocking would delay margin recovery and strain dividend coverage.
  \item \textbf{LTSA unwind}: Accelerated refunds or renegotiations could reduce deferred revenue and put further pressure on cash while customers retain take-or-pay leverage.
  \item \textbf{Leverage and payouts}: Maintaining nearly $1B annual dividends plus pref coupons with sub-$1B operating cash leaves little cushion if utilization remains low.
  \item \textbf{Tax and legal exposure}: IRS transfer-pricing (FY2007--2016), Malaysian assessment (up to \$410M), and German ORIP disputes (up to \$92M) could trigger sizable cash settlements and higher effective tax rates.
  \item \textbf{Execution on cost actions}: Failure to migrate assembly/test in-house or delay in closing low-utilization fabs would keep unabsorbed overhead elevated.
\end{itemize}

\section*{Scenario outlook}
\begin{table}[h!]
  \centering
  \begin{tabular}{p{2cm}p{6.6cm}cc}
    \toprule
    Scenario & Key assumptions & FY2026 revenue ($\!$B) & FY2026 EPS (USD) \\
    \midrule
    Bear & Industrial/auto demand flat, LTSA refunds continue, gross margin stuck near 55\%, dividends maintained; leverage rises toward covenants. & 4.3 & 1.10 \\
    Base & Inventory correction ends mid-2026, utilization improves, gross margin back to 58\%, modest opex relief, tax rate normalizes high teens. & 4.6 & 1.44 \\
    Bull & Accelerated restocking plus CHIPS support enables revenue snapback and 60\%+ gross margin; PolarFire \\ automotive sockets add upside, share count drifts lower. & 5.4 & 2.50 \\
    \bottomrule
  \end{tabular}
  \caption{Scenario guardrails anchored on FY2025 actuals, FY2026 run-rate metrics, and consensus inputs (Semiconductor List.xlsx: FY1 revenue $4.55$B, FY2 EPS $2.50$).}
\end{table}

\section*{Action items}
\begin{itemize}
  \item Track distributor inventory days, LTSA refund disclosures, and backlog commentary each quarter to gauge timing of volume recovery.
  \item Monitor cash/credit metrics (commercial paper outstandings, revolver usage, leverage covenant headroom) versus dividend commitments.
  \item Follow CHIPS Act negotiations and any incremental domestic capex plans that could unlock grants or tax credits.
  \item Watch analog and MCU booking trends by end market (industrial, auto, comms, aero/defense) along with PolarFire design wins.
  \item Update valuation comps (PS NTM 5.8x, PE FY2026 38x, dividend yield 3.3\%, beta 1.47) after each earnings release relative to analog peers (TXN, ADI) and MCU-focused rivals (NXP, Renesas).
\end{itemize}

\section*{Appendix: data notes}
\begin{itemize}
  \item Figures sourced from Microchip FY2025 Form 10-K and Q2 FY2026 Form 10-Q text extractions within this workspace; dollar amounts converted to USD billions and rounded to two decimals unless noted.
  \item Free cash flow defined as GAAP operating cash flow minus capital expenditures; margins calculated from GAAP financials.
  \item Valuation metrics, beta, EPS/revenue forecasts, dividend yield, leverage, volatility, and short interest pulled from ``Semiconductor List.xlsx'' dated November 2025.
  \item Inventory days, distributor mix, LTSA commentary, and CHIPS grant details quoted directly from management discussions in the referenced filings.
\end{itemize}

\end{document}
