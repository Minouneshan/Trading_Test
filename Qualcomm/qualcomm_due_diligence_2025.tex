\documentclass[11pt]{article}
\usepackage[margin=1in]{geometry}
\usepackage{booktabs}
\usepackage{array}
\usepackage{multirow}
\usepackage{hyperref}
\usepackage{enumitem}
\setlist{leftmargin=1.6em, itemsep=0.3em}
\hypersetup{
  colorlinks=true,
  linkcolor=black,
  urlcolor=blue
}

\begin{document}
\begin{center}
  {\LARGE Qualcomm FY2025 + Run-Rate Due Diligence Brief}\\[4pt]
  {\large 23 November 2025}
\end{center}

\section*{Executive takeaways}
\begin{itemize}
  \item FY2025 revenue reached $44.3$B (+14\% YoY) with equipment \& services up 15\% and licensing up 4\%; QCT carried 87\% of sales and broadened beyond handsets as automotive \& IoT combined for $10.6$B (Qualcomm Form 10-K).
  \item Operating income expanded to $12.4$B (27.9\% margin) on disciplined opex even as GAAP net income fell to $5.5$B (12.5\% margin) because of a $7.1$B tax charge tied to audit settlements and repatriation adjustments.
  \item Cash generation remained elite: $14.0$B operating cash (32\% of revenue), $1.2$B capex, and $12.8$B free cash flow funded $8.8$B buybacks plus $3.8$B dividends while net debt stayed modest.
  \item Q3 FY2025 delivered $10.4$B revenue (+10\% YoY) and $2.7$B operating income with QCT EBT margins holding ~30\%; nine-month operating cash already $11$B despite elevated working-capital needs (June 29, 2025 Form 10-Q).
  \item Valuation screens inexpensive relative to quality: shares at ~$174$ imply 4.1x NTM sales, 14.3x NTM EPS, 2.0\% dividend yield, and $187$B market cap versus peers trading 25--60x earnings on similar or lower cash conversion (Semiconductor List.xlsx).
\end{itemize}

\section*{Source corpus}
\begin{itemize}
  \item Qualcomm Incorporated Form 10-K for FY2025 (filed 5 November 2025) --- Excel extracts for consolidated statements, segment detail, capex, cash flow, and balance sheet.
  \item Qualcomm Incorporated Form 10-Q for Q3 FY2025 (quarter ended 29 June 2025, filed 30 July 2025) --- quarterly operations, nine-month cash flow, and segment commentary.
  \item ``Semiconductor List.xlsx'' (November 2025 refresh) --- market data, consensus revenue/EPS, valuation multiples, beta, leverage, volatility, and sentiment statistics.
\end{itemize}

\section*{QCT/QTL portfolio map}
\begin{itemize}
  \item \textbf{QCT (Qualcomm CDMA Technologies)}: Handsets (Snapdragon 8/7 platforms, RF front-end, modems), Automotive (Digital Chassis, Ride Flex, telematics), IoT (industrial edge, XR, Wi-Fi 7, PC compute modules).
  \item \textbf{QTL (Qualcomm Technology Licensing)}: Cellular SEPs, application processor IP, and systems patents monetized across Android, Apple, and IoT OEMs; structurally high-margin cash generator.
  \item \textbf{Capital-light model}: Fabless execution with multi-source foundry partners (TSMC/Samsung), limited capex needs, and heavy R\&D investment focused on modem-to-edge AI compute stacks.
\end{itemize}

\section*{FY2025 scoreboard}
\begin{table}[h!]
  \centering
  \begin{tabular}{lccc}
    \toprule
    Metric (USD in billions) & FY2025 & FY2024 & YoY \\
    \midrule
    Revenue & 44.3 & 39.0 & +13.6\% \\
    Gross margin & 24.5 & 21.9 & +12.0\% \\
    Operating income & 12.4 & 10.1 & +22.7\% \\
    Net income & 5.5 & 10.1 & -45.3\% \\
    Operating cash flow & 14.0 & 12.2 & +14.8\% \\
    Capital expenditures & 1.2 & 1.0 & +14.5\% \\
    Free cash flow & 12.8 & 11.2 & +14.5\% \\
    R\&D expense & 9.0 & 8.9 & +1.7\% \\
    Share repurchases & 8.8 & 4.1 & +113\% \\
    Gross margin & 55.4\% & 56.2\% & -80 bps \\
    Operating margin & 27.9\% & 25.8\% & +210 bps \\
    Net margin & 12.5\% & 26.0\% & -1,350 bps \\
    \bottomrule
  \end{tabular}
  \caption{Qualcomm FY2025 consolidated results (millions converted to USD billions). Free cash flow defined as operating cash flow minus capital expenditures.}
\end{table}

\subsection*{Observations}
\begin{itemize}
  \item Margin profile still healthy despite handset volatility; the apparent EPS collapse is purely tax driven with no deterioration in core profitability.
  \item Working capital provided $1.4$B of cash (inventory modestly higher for new Snapdragon ramps, offset by receivables collections and lower accrued comp).
  \item R\&D intensity of 20\% reflects investment in custom Oryon CPU cores, generative AI software (AI Hub), and automotive silicon; SG\&A leverage kept operating margin expanding 210 bps.
  \item Buybacks + dividends returned $12.6$B (90\% of free cash flow), showing confidence in medium-term demand and patent cash streams.
\end{itemize}

\section*{Segment and mix dynamics}
\subsection*{FY2025 QCT revenue detail}
\begin{table}[h!]
  \centering
  \begin{tabular}{lccc}
    \toprule
    Category & Revenue ($\!$B) & Mix & YoY \\
    \midrule
    Handsets & 27.8 & 72.5\% & +12\% \\
    Automotive & 4.0 & 10.3\% & +36\% \\
    IoT & 6.6 & 17.2\% & +22\% \\
    Total QCT & 38.4 & 86.7\% & +15.6\% \\
    QTL licensing & 6.4 & 13.3\% & +4\% \\
    \bottomrule
  \end{tabular}
  \caption{Segment disclosures from the FY2025 Form 10-K; mix based on consolidated revenue.}
\end{table}

\subsection*{Key mix takeaways}
\begin{itemize}
  \item \textbf{Automotive pipeline}: Revenue $4.0$B with EBT margin near 30\%; design-win backlog now $45+$B with BMW, GM, Mercedes, Hyundai, and Chinese EV OEMs adopting Snapdragon Ride Flex and Car-to-Cloud.
  \item \textbf{IoT normalization}: Industrial/edge computing, XR, and Wi-Fi 7 CPE sustained $6.6$B revenue while low-end consumer IoT remains inventory managed.
  \item \textbf{Licensing durability}: QTL EBT $5.2$B (81\% margin) even with Apple royalties stepping down; QCOM maintained near-100\% compliance across major 5G OEMs.
\end{itemize}

\section*{Latest quarterly pulse (Q3 FY2025)}
\begin{itemize}
  \item Revenue $10.365$B (+10\% YoY) with operating income $2.762$B (26.6\% margin) and net income $2.666$B (diluted EPS $2.43$); nine-month revenue $33.0$B (+15\%).
  \item QCT posted $8.993$B quarterly revenue (Handsets $6.33$B, Automotive $0.98$B, IoT $1.68$B) and $2.67$B segment EBT (29.7\% margin); QTL added $1.318$B revenue and $0.94$B EBT.
  \item Operating cash flow for the first nine months reached $11.0$B with $0.86$B capex and $6.4$B buybacks; quarter-end liquidity was $9.9$B cash + $4.3$B marketable securities against $15.0$B total debt.
  \item Management flagged broad-based Snapdragon 8 Gen 4 adoption (Samsung, vivo, Xiaomi), first Ride Flex SOPs at BMW/GM mid-2025, and strengthening demand for Windows-on-ARM AI PCs and industrial edge gateways.
\end{itemize}

\section*{Balance sheet, liquidity, and capital returns}
\begin{itemize}
  \item \textbf{Liquidity}: Cash $5.5$B + restricted cash $2.3$B + short-term securities $4.6$B = $12.5$B; revolving credit capacity remains undrawn.
  \item \textbf{Leverage}: Long-term debt $14.8$B with maturities laddered through 2053; debt-to-equity 0.70x and net debt roughly $2.3$B after including restricted cash.
  \item \textbf{Working capital}: Receivables $4.3$B, inventories $6.5$B, other current assets $2.4$B vs. current liabilities $9.1$B --- a 2.8x current ratio supporting advanced-node prepayments.
  \item \textbf{Cash return policy}: $0.80$ quarterly dividend ($3.20$ annualized, 2.0\% yield) plus opportunistic buybacks; FY2025 weighted-average repurchase price ~$189$.
  \item \textbf{Capex needs}: Fabless structure caps annual capex near $1$--$1.5$B (mostly lab/test and design tools), keeping free cash flow conversion above 90\%.
\end{itemize}

\section*{Strategic themes and catalysts}
\begin{enumerate}
  \item \textbf{Edge/on-device AI monetization}: Snapdragon X Elite PCs, 8 Gen 4 phones, and AI Hub software create CPU/GPU/NPU attach while limiting dependence on handset unit growth.
  \item \textbf{Automotive flywheel}: Digital Chassis backlog exceeds $45$B with zonal controllers, infotainment, ADAS, and connectivity; blended dollar content per vehicle continues to climb.
  \item \textbf{Diversified revenue streams}: Licensing cash plus automotive/IoT exposure reduces reliance on premium Android cycles and smooths free cash flow.
  \item \textbf{Custom silicon + foundry partnerships}: Collaboration with TSMC/Nuvia-derived CPU roadmap aims to capture Windows-on-ARM share and potential datacenter accelerator sockets.
  \item \textbf{Capital deployment}: High free cash flow enables continued buybacks/dividends while funding tuck-in AI software \& RF front-end acquisitions.
\end{enumerate}

\section*{Risks and watch items}
\begin{itemize}
  \item \textbf{Handset unit volatility}: Android flagship sell-through remains macro sensitive; a weak 2026 refresh could stall QCT growth.
  \item \textbf{Regulatory/tax uncertainty}: Ongoing global tax audits and potential changes to SEP licensing frameworks create earnings volatility similar to FY2025.
  \item \textbf{Apple modem exposure}: Any unexpected end to the Apple modem extension (currently through 2027) would trim handset revenue until standalone RF \& application processors backfill.
  \item \textbf{Automotive execution}: Software integration (Ride Flex, safety certifications) must stay on schedule to convert the $45$B design-win pipeline into revenue.
  \item \textbf{Geopolitical/export controls}: Restrictions on advanced SoCs or RF modules to China, or broader supply-chain disruptions, could pressure both QCT shipments and royalty collections.
\end{itemize}

\section*{Scenario outlook}
\begin{table}[h!]
  \centering
  \begin{tabular}{p{2cm}p{6.6cm}cc}
    \toprule
    Scenario & Key assumptions & FY2026 revenue ($\!$B) & FY2026 EPS (USD) \\
    \midrule
    Bear & Android volumes flat, automotive SOPs slip a year, and regulators force incremental royalty rebates; buybacks slow. & 43 & 11.0 \\
    Base & Stable Android mix, Snapdragon X Elite PCs reach mid-single-digit PC share, automotive backlog converts on schedule, tax rate normalizes to high-teens. & 47 & 12.4 \\
    Bull & Accelerated on-device AI adoption plus incremental Apple/PC wins, automotive pipeline pull-ins, and continued buybacks shrink share count 3\%. & 50 & 13.2 \\
    \bottomrule
  \end{tabular}
  \caption{Scenario guardrails anchored on FY2025 actuals, FY2025 Q3 run-rate, and consensus inputs (Semiconductor List.xlsx: FY1 revenue $45.6$B, FY2 EPS $12.43$).}
\end{table}

\section*{Action items}
\begin{itemize}
  \item Track Snapdragon X Elite/Oryon PC design wins (Microsoft, Lenovo, HP) and software ecosystem benchmarks to validate PC share assumptions.
  \item Monitor automotive SOP milestones (BMW Neue Klasse, GM Ultra Cruise, Chinese EV launches) and attach-rate announcements.
  \item Watch handset channel inventory, Android flagship ASPs, and RF front-end share to gauge FY2026 revenue glide path.
  \item Follow global tax/SEP litigation progress to anticipate potential reversals of the FY2025 one-time tax charge.
  \item Refresh valuation comps (PS NTM 4.1x, PE NTM 14.3x, beta 1.10) after each earnings print relative to analog/mixed-signal peers and AI CPU/GPU leaders.
\end{itemize}

\section*{Appendix: data notes}
\begin{itemize}
  \item Figures sourced from Qualcomm FY2025 Form 10-K and Q3 FY2025 Form 10-Q spreadsheets bundled in the workspace; amounts converted from millions to USD billions and rounded to one decimal unless noted.
  \item Free cash flow defined as operating cash flow minus capital expenditures; margins and mixes computed from GAAP figures.
  \item Valuation metrics, beta, EPS/revenue growth, dividend yield, and short interest pulled from ``Semiconductor List.xlsx'' (November 2025).
  \item Segment abbreviations follow Qualcomm disclosures: QCT (Qualcomm CDMA Technologies) and QTL (Qualcomm Technology Licensing).
\end{itemize}

\end{document}
