\documentclass[11pt]{article}
\usepackage[margin=1in]{geometry}
\usepackage{booktabs}
\usepackage{array}
\usepackage{multirow}
\usepackage{hyperref}
\usepackage{enumitem}
\setlist{leftmargin=1.6em, itemsep=0.3em}
\hypersetup{
  colorlinks=true,
  linkcolor=black,
  urlcolor=blue
}

\begin{document}
\begin{center}
  {\LARGE Intel FY2024 + FY2025 Run-Rate Due Diligence Brief}\\[4pt]
  {\large 22 November 2025}
\end{center}

\section*{Executive takeaways}
\begin{itemize}
  \item Intel finished FY2024 at $53.1$B in revenue (-2\% YoY) with gross margin compressed to 32.7\% and a GAAP operating loss of $11.7$B as IDM 2.0 modernization, inventory reserves, and restructuring charges totaled $7$B (FY2024 Form 10-K).
  \item Client Computing returned to modest growth ($30.3$B, +3.5\%) on AI PC enablement, while Data Center \& AI ($12.8$B, +1.4\%) leaned on Gaudi and Granite Rapids sampling; Network \& Edge stayed flat, and Intel Foundry posted a $13.4$B operating loss as 18A/20A fabs ramped below absorption targets (Operating Segments note).
  \item Cash from operations held at $8.3$B despite losses, but $23.9$B of capex kept free cash flow negative $15.6$B; Intel is relying on $10$B+ of CHIPS/EU subsidies, preferred equity, and strategic partnerships (U.S. DOE, Brookfield) to fund IDM 2.0.
  \item Balance sheet liquidity remains solid (cash + short-term investments $22$B), yet debt sits at $46.3$B and inventories climbed to $12.2$B; cost controls, factory loading, and working-capital discipline are critical in 2025.
  \item Q1 FY2025 revenue was $12.7$B (-0.4\% YoY): DCAI grew 7.8\%, NEX +46\%, and Foundry +7\% while CCG fell 8\% awaiting AI PC launches; GAAP EPS stayed negative (-\$0.09) as foundry absorption and restructuring charges persisted (March 29, 2025 Form 10-Q).
\end{itemize}

\section*{Source corpus}
\begin{itemize}
  \item Intel Corporation Form 10-K for the fiscal year ended December 28, 2024 (filed February 2025).
  \item Intel Corporation Form 10-Q for the quarter ended March 29, 2025 (filed April 2025).
  \item Intel Corporation Form 10-Q for the quarter ended September 28, 2025 (filed November 2025, PDF reference for outlook/color).
  \item ``Semiconductor List.xlsx'' (updated November 2025) for peer growth, valuation, and profitability benchmarks.
\end{itemize}

\section*{IDM 2.0 transformation map}
\begin{itemize}
  \item \textbf{Foundry services (IFS)}: 18A/20A process portfolio, advanced packaging (Foveros, EMIB), CHIPS-funded mega fabs in Arizona/Ohio/Germany, and external customers (MediaTek, Amazon, U.S. DoD) alongside internal wafers.
  \item \textbf{Client (CCG)}: Meteor Lake/Arrow Lake, Lunar Lake ultra-mobile, AI PC platforms with on-die NPU (Copilot+), and Evo/Commercial vPro device stack.
  \item \textbf{Data Center \& AI (DCAI)}: Xeon 6 (E-core/Sierra Forest, P-core/Granite Rapids), Gaudi 2/3 accelerators, Falcon Shores GPU, Ethernet switching, and Habana software stack.
  \item \textbf{Network \& Edge (NEX)}: 5G vRAN, Time Coordinated Computing, industrial control, Movidius/Altera FPGA edge accelerators, private wireless partnerships.
  \item \textbf{Software and services}: OpenVINO, oneAPI, AI PC frameworks, internal foundry PDK/IP catalog.
  \item \textbf{Capital partnerships}: Brookfield/OH STID financing, CHIPS Act grants/loans, EU/German subsidies, and customer co-invest (e.g., Microsoft for IFS packaging).
\end{itemize}

\section*{FY2024 scoreboard}
\begin{table}[h!]
  \centering
  \begin{tabular}{lccc}
    \toprule
    Metric (USD in billions) & FY2024 & FY2023 & YoY \\
    \midrule
    Revenue & 53.1 & 54.2 & -2.1\% \\
    Gross margin & 17.3 & 21.7 & -20.1\% \\
    Operating income & -11.7 & 0.1 & NM \\
    Net income & -19.2 & 1.7 & NM \\
    Operating cash flow & 8.3 & 11.5 & -27.7\% \\
    Capital expenditures & 23.9 & 25.8 & -7.0\% \\
    Free cash flow & -15.6 & -14.3 & Worse \\
    R\&D expense & 16.5 & 16.0 & +3.1\% \\
  Restructuring \& other & 7.0 & (0.1) & NM \\
    Gross margin & 32.7\% & 40.0\% & -730 bps \\
    Operating margin & -22.0\% & 0.2\% & NM \\
    Net margin & -36.2\% & 3.1\% & NM \\
    \bottomrule
  \end{tabular}
  \caption{Consolidated results derived from Intel FY2024 Form 10-K (amounts converted from millions).}
\end{table}

\subsection*{Observations}
\begin{itemize}
  \item $6.97$B of restructuring/product charges plus higher start-up depreciation erased the modest gross profit base, yielding full-year operating losses despite positive operating cash flow.
  \item R\&D intensity (31\% of revenue) reflects concurrent CPU, GPU, accelerator, and process-node roadmaps; Intel reiterated \$15B+ annual R\&D through 2026.
  \item Free cash flow stayed deeply negative until IFS loads its megafabs; management expects external wafers and subsidies to close the gap by 2026.
\end{itemize}

\section*{Segment and mix dynamics}
\subsection*{FY2024 segment revenue}
\begin{table}[h!]
  \centering
  \begin{tabular}{lccc}
    \toprule
    Segment & Revenue (\$B) & Mix & YoY \\
    \midrule
    Client Computing Group & 30.3 & 57\% & +3.5\% \\
  Data Center \& AI & 12.8 & 24\% & +1.4\% \\
  Network \& Edge & 5.8 & 11\% & +1.2\% \\
    Intel Foundry (IFS) & 17.5 & 33\% & -7.2\% \\
    All other/eliminations & -13.3 & -25\% & Improved \\
    \bottomrule
  \end{tabular}
  \caption{Reportable segment breakout per FY2024 Form 10-K, Note 3. Segment mix shown versus consolidated revenue.}
\end{table}

\subsection*{Latest quarterly pulse (Q1 FY2025)}
\begin{itemize}
  \item Revenue $12.7$B (-0.4\% YoY); CCG $7.63$B (-7.8\%), DCAI $4.13$B (+7.8\%), NEX $0.94$B (+46\%), Intel Foundry $4.67$B (+7.1\%).
  \item Gross margin $5.5$B (43.4\%) vs. $5.2$B (41.0\%) prior year; operating loss -$1.1$B as Foundry GM stayed -21\%.
  \item Cash from operations $1.1$B; capex $4.8$B keeps quarterly free cash flow near -$3.7$B despite initial CHIPS inflows.
\end{itemize}

\section*{Balance sheet, liquidity, and capital intensity}
\begin{itemize}
  \item \textbf{Liquidity}: Cash $8.2$B plus short-term investments $13.8$B provide $22$B of liquidity; Intel also has $10$B+ undrawn green/CHIPS-linked credit facilities.
  \item \textbf{Leverage}: Total debt $46.3$B (including $3.7$B short-term) vs. equity $99.3$B; rating agencies keep Intel at A-/Baa1 contingent on subsidy execution and negative free cash flow narrowing.
  \item \textbf{Working capital}: Inventories $12.2$B (days inventory \\textasciitilde{}190) as Intel stages EUV nodes; accounts payable $12.6$B reflects tool deliveries and co-invest structures.
  \item \textbf{Capex pipeline}: 2025 capex guide remains ``low-$20$B'' even after Brookfield partnership; subsidies (US CHIPS, EU/Germany, Israel) expected to reimburse $5$--$7$B annually once milestones hit.
  \item \textbf{Capital returns}: Dividend remains suspended; buybacks paused since 2022 to prioritize fab funding.
\end{itemize}

\section*{Strategic themes and catalysts}
\begin{enumerate}
  \item \textbf{Foundry scale-up}: Landing anchor customers (DoD, AWS, MediaTek, Microsoft) on 18A/20A and Foveros Direct is key to absorbing $13$B+ of annual fixed cost.
  \item \textbf{AI PC refresh}: Copilot+ premium notebooks, Lunar Lake, and Arrow Lake are the primary levers to stabilize CCG mix and margin.
  \item \textbf{Gaudi/Falcon Shores ramp}: Offering a lower-cost accelerator alternative while Falcon Shores unifies GPU + vector IP in 2026; need proof points (Azure, AWS) to validate software stack.
  \item \textbf{Cost resets}: $10$B cost-out program through 2025 targets opex and COGS; monitoring factory loading, EUV utilization, and headcount actions.
  \item \textbf{Government/partner funding}: Execution on CHIPS grants, DOE loans, and customer co-invest (Brookfield, Apollo, PC OEMs) determines whether capex stays sustainable.
\end{enumerate}

\section*{Risks and watch items}
\begin{itemize}
  \item \textbf{Foundry economics}: Prolonged negative gross margin at IFS would pressure liquidity before subsidies arrive.
  \item \textbf{PC cycle timing}: If AI PC uptake slips beyond holiday 2025, CCG mix could revert to low-end Chromebooks with weaker ASPs.
  \item \textbf{Accelerator competition}: NVIDIA Blackwell and custom silicon (Amazon, Google) may limit Gaudi attach unless Intel rapidly expands software support.
  \item \textbf{Supply chain execution}: 18A/20A yield learning, tool install bottlenecks, or HVM delays could jeopardize both internal Xeon 6 and external foundry commitments.
  \item \textbf{Macro/geo}: Export controls on China data centers (where Intel still has share) and potential delays in EU subsidy disbursements remain key swing factors.
\end{itemize}

\section*{Scenario outlook}
\begin{table}[h!]
  \centering
  \begin{tabular}{p{2cm}p{6.4cm}cc}
    \toprule
    Scenario & Key assumptions & FY2025 revenue (\$B) & FY2025 EPS (USD) \\
    \midrule
    Bear & AI PC adoption slips to late 2026, Foundry GM stays below -15\%, CHIPS cash pushes to 2026, DCAI share losses to NVIDIA Gaudi alternatives. & 50 & -0.40 \\
    Base & AI PCs reach mid-teens PC mix, DCAI grows high-single digits on Xeon 6 + Gaudi, Foundry losses narrow with first external 18A wafers and CHIPS reimbursements. & 55 & 0.15 \\
    Bull & CHIPS funds arrive early, Microsoft/DoD lock multi-year IFS capacity, Gaudi wins two hyperscale training clusters, PC refresh drives double-digit CCG growth. & 58 & 0.60 \\
    \bottomrule
  \end{tabular}
  \caption{Scenario guardrails anchored on FY2024 actuals, Q1 FY2025 run-rate, and peer estimates from Semiconductor List.xlsx.}
\end{table}

\section*{Action items}
\begin{itemize}
  \item Monitor CHIPS Act grant/loan milestones (U.S. and EU) and related cash inflows to gauge capex funding risk.
  \item Track OEM launches of Copilot+ and AI PC skus (HP, Dell, Lenovo) plus attach rates for Intel's NPU roadmap.
  \item Compare Gaudi win announcements versus NVIDIA Blackwell/Hopper deployments to judge DCAI upside.
  \item Review quarterly disclosures on IFS external revenue, backlog, and packaging utilization to assess fixed-cost absorption.
  \item Refresh valuation comps (PS NTM 3.2x, PE NTM 63.8x, beta 1.36) against AMD, NVIDIA, Broadcom, Qualcomm once FY2025 guide updates arrive.
\end{itemize}

\section*{Appendix: data notes}
\begin{itemize}
  \item Figures sourced from Intel FY2024 Form 10-K and Q1 FY2025 Form 10-Q; amounts reported in millions converted to USD billions and rounded to one decimal unless noted.
  \item Free cash flow defined as net cash provided by operating activities minus additions to property, plant, and equipment.
  \item Segment percentages calculated against consolidated revenue; negative ``all other" reflects corporate allocations and inter-segment eliminations.
  \item Valuation data (market cap, PS/PE, beta, net margin) taken from ``Semiconductor List.xlsx" (November 2025 refresh).
\end{itemize}

\end{document}
